\documentclass[ngerman,xcolor={table,dvipsnames},smaller,compress,hyperref={bookmarks,colorlinks}]{beamer}%,handout

\usepackage{url}
\usepackage{listings}
\usepackage[latin9]{inputenc}
\usepackage{textcomp}
%\usepackage{auto-pst-pdf}
\usepackage{graphicx}
\usepackage{pstricks}
\usepackage{pst-node}
\usepackage{pst-uml}
\usepackage{pst-tree}
\usepackage{pst-blur}

\usepackage{amsfonts}
\usepackage{amssymb}
\usepackage{amsmath}
\usepackage{pifont}

% for multipage tables (xtab or longtable
\usepackage{longtable}
\usepackage{lscape}
\usepackage{booktabs}
\usepackage[tableposition=top]{caption}

\title{CMDB Driven by Perl}
\author{Jens Rehsack}
\institute[Niederrhein.PM]{Niederrhein Perl Mongers}
\date{Niederrhein.PM Meeting June 2017}

\usetheme[secheader]{Boadilla}
\setbeamertemplate{navigation symbols}{}

\newcommand{\perlfilename}[1]{{\color{cyan}{\textit{\begingroup \urlstyle{sf}\Url{#1}}}}}
\newcommand{\xsfilename}[1]{{\color{olive}{\textit{\begingroup \urlstyle{sf}\Url{#1}}}}}
\newcommand{\hfilename}[1]{{\color{olive}{\textit{\begingroup \urlstyle{sf}\Url{#1}}}}}
\newcommand{\cfilename}[1]{{\color{magenta}{\textit{\begingroup \urlstyle{sf}\Url{#1}}}}}
\newcommand{\variable}[1]{{\color{violet}{\textsf{#1}}}}
\newcommand{\variablew}[1]{{{\textsf{#1}}}}
\newcommand{\command}[1]{'\texttt{#1}'}

\def\signed #1{{\leavevmode\unskip\nobreak\hfil\penalty50\hskip2em
   \hbox{}\nobreak\hfil(#1)%
   \parfillskip=0pt \finalhyphendemerits=0 \endgraf}}
 
\newsavebox\mybox
\newenvironment{aquote}[1]
   {\savebox\mybox{#1}\begin{quotation}}
   {\signed{\usebox\mybox}\end{quotation}}

\makeatletter
\makeatother

\begin{document}

\psset{angleA=90,angleB=-90}
\lstset{language=Perl,
        basicstyle=\ttfamily,
        keywordstyle=\color{Maroon},
        commentstyle=\color{Blue}, 
        stringstyle=\color{Green},
        showstringspaces=false}

\AtBeginPart{\begin{frame}<beamer> \frametitle{Overview} \partpage \tableofcontents[current] \end{frame}}

\frame{\maketitle}

\part{Introduction}

\section{Introduction}

\subsection{Motivation}

\begin{frame}[fragile]{Motivation}
\begin{block}{Progress}
\begin{aquote}{Robert A. Heinlein}
Progress isn't made by early risers. It's made by lazy men trying to find easier ways to do something. \\
\end{aquote}
\end{block}
\end{frame}

\begin{frame}[fragile]{Motivation}
\begin{block}{Efficiency}
\begin{aquote}{Elizabeth Mattijsen}
Business success is because of Perl. It enables us to deliver right solutions in days instead of months. \\
\end{aquote}
\end{block}
\end{frame}

\begin{frame}[fragile]{Goal}
\begin{block}{Automation}
Full flavoured systems management
\begin{itemize}
\item<2-> Installation without Administrator interaction
\item<3-> Ensured system state by actual-theoretical comparison
\item<4-> Faster reaction in emergency cases by organized component moving
\item<5-> Control sensors and alarming
\end{itemize}
\end{block}
\end{frame}

\part{Challenge}

\section{Beginning}

\subsection{Taking over}

\begin{frame}[fragile]{Begin with reporting}
\begin{block}{Begin with reporting}
In the beginning was the (Installation-)Report
\begin{itemize}
\item<2-> Technical Sales defined an XML Document for Change Requests and Status Reports
\item<3-> Document Definition lacks entity-relations
\item<4-> Document Definition misses technolgy requirements
\item<5-> Based on work of forerunner a 70\% solution could be delivered
\end{itemize}
\end{block}
\end{frame}

\subsection{After reporting}

\begin{frame}[fragile]{From reporting to \ldots}
\begin{block}{Mind the goal}
\begin{aquote}{Lewis Carroll}
Alice: Would you tell me, please, which way I ought to go from here? \\
The Cheshire Cat: That depends a good deal on where you want to get to. \\
\end{aquote}
\end{block}
\end{frame}

\subsection{Baby Steps}

\begin{frame}[fragile]{Baby Steps}
\begin{block}<1->{Improve knowledge}
Based on identified issues the first goal had to be to identify all entities and their relations together
\end{block}
\begin{block}<2->{Surrounded}
Problem: The entire platform was completely unstructured
\end{block}
\end{frame}

\subsection{Mine Sweeper}

\begin{frame}[fragile]{Baby Steps}
\begin{block}{Multiple Beginnings}
\begin{itemize}
\item The already know ''(Installation-)Report''
\item<2-> Platform Snapshot (SCM Repository of selected configuration files)
\item<3-> Puppet Classes (without \textit{Hiera}) mixed of \textit{Configuration Items} and prepared configuration files (unsupervised)
\end{itemize}
\end{block}
\end{frame}

\part{Sweat}

\section{Concerns}

\subsection{Identifying}

\begin{frame}[fragile]{Concerns}
\begin{block}{Rough}
\begin{itemize}
\item Collecting of platform parameters (to access them in SQL'ish way)
\item<2-> Identify coherences of \textit{Configuration Items}
\item<3-> Define a data model for \textit{Configuration Management Database}
\item<4-> Define technical requirements of our \texttt{CMDB} incarnation
\end{itemize}
\end{block}
\end{frame}

\begin{frame}[fragile]{Concerns}
\begin{block}{Techical}
\begin{itemize}
\item validity of \texttt{CI}'s
\item<2-> limits of our \texttt{CI}'s
\item<3-> data ownership of \texttt{CI}'s
\item<4-> methods to access \texttt{CMDB}
\item<5-> permission management
\end{itemize}
\end{block}
\end{frame}

\subsection{Separation}

\begin{frame}[fragile]{}
\begin{block}{}

\resizebox{\textwidth}{!}{
\begin{pspicture}(-7,-5)(7,5)%\psgrid
     % 
     \uput{4}[90](0,0){\rnode{platform}{\umlClass{\underline{:platform}}{}}}
     \uncover<2->{\uput{4}[18](0,0){\rnode{snapshot}{\umlClass{\underline{:snapshot}}{}}}}
     \uncover<3->{\uput{4}[306](0,0){\rnode{scandb}{\umlClass{\underline{:scan\_db}}{}}}}
     \uncover<4->{\uput{4}[234](0,0){\rnode{cmdb}{\umlClass{\underline{:cmdb}}{}}}}
     \uncover<5->{\uput{4}[162](0,0){\rnode{hiera}{\umlClass{\underline{:hiera}}{}}}}
     \uncover<8->{\rput(-6,-4){\rnode{report}{\umlClass{\underline{:report}}{}}}}

     \uncover<2->{\ncarc[arcangle=30,linewidth=0.5pt,linestyle=solid]{->}{platform}{snapshot}\naput*[Ynodesep=0.5]{collect}}
     \uncover<3->{\ncarc[arcangle=30,linewidth=0.5pt,linestyle=solid]{->}{snapshot}{scandb}\naput*[Ynodesep=0.5]{scan}}
     \uncover<4->{\ncarc[arcangle=30,linewidth=0.5pt,linestyle=solid]{->}{scandb}{cmdb}\naput*[Ynodesep=0.5]{process-scan}}
     \uncover<5->{\ncarc[arcangle=30,linewidth=0.5pt,linestyle=solid]{->}{cmdb}{hiera}\naput*[Ynodesep=0.5]{yaml-gen}}
     \uncover<6->{\ncarc[arcangle=30,linewidth=0.5pt,linestyle=solid]{->}{hiera}{platform}\naput*[Ynodesep=0.5]{puppet run}}
     \uncover<7->{\nccircle[angleA=-40,angleB=90,arm=.8,linearc=.2,linewidth=0.5pt,linestyle=solid]{->}{cmdb}{1cm}\naput*[Ynodesep=0.5]{CRQ}}
     \uncover<8->{\ncarc[arcangle=30,linewidth=0.5pt,linestyle=solid]{->}{cmdb}{report}\naput*[Ynodesep=0.5]{export}}

\end{pspicture}
}%end resizebox

\end{block}
\end{frame}

\section{Tuck In!}

\subsection{MI:5}

\begin{frame}[fragile]{Where do I begin}
\begin{block}{To write the workflow how great Perl 5 can be}
\begin{itemize}
\item The project was in a state where a developer created a particular Report based on the existing snapshot.
\item<2-> This solution did not maintain an abstraction layer for gathered data - every time when the report needs an extension, an end-to-end (snapshot to XML-Tag) enhancement had to be created.
\item<3-> Changes shall be deployed from the same report format as installations are reported.
\item<4-> We have to be able to say at any moment what is operated on the platform.
\end{itemize}
\end{block}
\end{frame}

\begin{frame}[fragile]{Impossible Things}
\begin{block}{Impossible Things}
\begin{aquote}{Lewis Carroll}
Alice laughed. ''There's no use trying,'' she said: ''one can't believe impossible things.'' \\
\hfil \\
''I daresay you haven't had much practice,'' said the Queen. ''When I was your age,
I always did it for half-an-hour a day. Why, sometimes I've believed as many as six
impossible things before breakfast.''
\end{aquote}
\end{block}
\end{frame}

\subsection{Home Improvement}

\begin{frame}[fragile]{The Fool with the Tool}
\begin{block}{Try again}
So we closed our eyes, took a deep breath (multiple times) and looked around for tools to store serialized data and read in structured way \ldots
\end{block}
\begin{block}<2->{Tool Time}
\begin{description}
\item<2->{MongoDB} Allows easy storing in any format - but lacks strucured querying dedicated entities (configuration items)
\item<3->{Data Files} Delegates relationship handling completely to business logic
\item<4->{AnyData2} Gotcha - allows reading most confusing stuff and could be queried in structured way
\end{description}
\end{block}
\end{frame}

\begin{frame}[fragile]{\ldots is still a Fool}
\begin{block}{Volatile Structure}
\begin{itemize}
\item Persist structured data using SQLite
\item<2-> Define a data model representing existing relations
\item<3-> Develop AnyData2::Format classes representing defined ER model
\item<4-> Develop MOP inside this AnyData2 instance to manage attributes vs. columns
\item<5-> Glue everything together using SQL
\end{itemize}
\uncover<6->{The entire ER model keeps a moving target}
\end{block}
\end{frame}

\subsection{Clean Picture}

\begin{frame}[fragile]{Abstraction Layer}
\begin{block}{\ldots of configured components}
\begin{itemize}
\item concentrate of the goal to know what is operated
\item<2-> depth first search over all compoenent configuration files
\item<3-> identify relationships (remember: the is no operation model at all)
\item<4-> clean up configuration when no reasonable relationships can be resolved or relationships are conflicting
\end{itemize}
\end{block}
\end{frame}

\subsection{Control}

\begin{frame}[fragile]{Moo in practice}
\begin{block}{Moo in practice}
It appears that the tools helped at safe IoT device updating are the same tools helping to coordinate CI determining:
\begin{description}
\item<2->{\texttt{MooX::Cmd}} helps separating concerns
\item<3->{\texttt{MooX::ConfigFromFile}} helps contribute ''divine wisdom''
\item<4->{\texttt{MooX::Log::Any}} feeds \texttt{DBIx::LogAny}
\end{description}
\end{block}
\end{frame}

\begin{frame}[fragile]{Moo in background}
\begin{block}{Moo in background}
\begin{itemize}
\item manage database connections based on concerns
\item<2-> manage \texttt{CI} structures based on relations
\item<3-> manage Web-API integration
\end{itemize}
\end{block}
\end{frame}

\subsection{Reflecting Relationships}

\begin{frame}[fragile]{}
\begin{block}{Crazy}
\begin{aquote}{Lewis Carroll}
I'm not crazy. My reality is just different than yours. \\
\hfil \\
The Cheshire Cat \\
\end{aquote}
\end{block}
\end{frame}

\begin{frame}[fragile]{}
\begin{block}{Harmonize Craziness}
\begin{itemize}
\item practically any administrator had a different background regarding to the platform components thus a different picture of their relationships
\item<2-> EPIC battles leads to common craziness
\item<3-> ER model analysis sessions uncover holes in picture
\end{itemize}
\end{block}
\end{frame}

\subsection{Meanwhile}

\begin{frame}[fragile]{}
\begin{block}{}
\begin{aquote}{Lewis Carroll}
March Hare: Have some wine. \\
(Alice looked all round the table, but there was nothing on it but tea.) \\
Alice: I don't see any wine. \\
\hfil \\
March Hare: There isn't any. \\
Alice: Then it wasn't very civil of you to offer it. \\
March Hare: It wasn't very civil of you to sit down without being invited. \\
\end{aquote}
\end{block}
\end{frame}

\begin{frame}[fragile]{New Goal}
\begin{block}{CentOS 5 ends it's maintenance}
\begin{itemize}
\item many of existing tools needs to be upgraded
\item<2-> Upgraded tools don't support existing hacks anymore
\item<3-> existing hacks must be replaced by a reasonable configuration structure
\item<4-> Same problem like the report format:
\begin{itemize}
\item<5-> neither the ER model of platform components nor issues of platform where known
\item<6-> neither cared about
\end{itemize}
\end{itemize}
\end{block}
\end{frame}

\begin{frame}[fragile]{Self Protection}
\begin{block}{Delegation}
We learned from mistakes of past:
\begin{itemize}
\item<2-> No responsibility taken for fill weird puppet templates
\item<3-> No external data will be managed
\item<4-> No precompiled/puzzled resources are prepared
\item[$\rightarrow$]<5-> ER model of \texttt{CMDB} is presented via \textit{RESTful API}
\end{itemize}
\end{block}
\end{frame}

\subsection{Merging Pictures}

\begin{frame}[fragile]{}
\begin{block}{Scan completed}
\begin{itemize}
\item Early implementation of above mentioned \textit{RESTful API} run against \textit{ScanDB}
\item<2-> \textit{ScanDB} represents just a view of the configuration snapshot
\item<3-> There is no future\uncover<4->{, neither past}
\item<5-> Time for \textit{CMDB} to enter the stage
\end{itemize}
\end{block}
\end{frame}

\begin{frame}[fragile]{}
\begin{block}{Future and Past}
\scriptsize
\begin{lstlisting}[language=SQL,inputencoding=latin9,escapeinside={(*@}{@*)}]
CREATE TABLE author
(
      author_id INTEGER PRIMARY KEY
    -- entity stuff
    , author_name VARCHAR(80) NOT NULL
    , author_birthday DATETIME NOT NULL
    , author_died_at DATETIME
    -- cmdb stuff
    , valid_from DATETIME NOT NULL
    , valid_to DATETIME
    , modified_at DATETIME NOT NULL
    , modified_by VARCHAR(32) NOT NULL
);
\end{lstlisting}
\end{block}
\end{frame}

\begin{frame}[fragile]{}
\begin{block}{Future and Past}
\scriptsize
\begin{lstlisting}[language=SQL,inputencoding=latin9,escapeinside={(*@}{@*)}]
CREATE TABLE book
(
      book_id INTEGER PRIMARY KEY
    -- entity stuff
    , book_name VARCHAR(80) NOT NULL
    , book_subtitle VARCHAR(80)
    , book_edition INTEGER
    -- cmdb stuff
    , valid_from DATETIME NOT NULL
    , valid_to DATETIME
    , modified_at DATETIME NOT NULL
    , modified_by VARCHAR(32) NOT NULL
);
\end{lstlisting}
\end{block}
\end{frame}

\begin{frame}[fragile]{}
\begin{block}{Future and Past}
\scriptsize
\begin{lstlisting}[language=SQL,inputencoding=latin9,escapeinside={(*@}{@*)}]
CREATE TABLE book_athors
(
      ba_id INTEGER PRIMARY KEY
    -- entity stuff
    , author_id INTEGER NOT NULL
    , book_id INTEGER NOT NULL
    -- cmdb stuff
    , valid_from DATETIME NOT NULL
    , valid_to DATETIME
    , modified_at DATETIME NOT NULL
    , modified_by VARCHAR(32) NOT NULL
    -- 
    , FOREIGN KEY (author_id) REFERENCES author(author_id) ON DELETE CASCADE
    , FOREIGN KEY (book_id) REFERENCES book(book_id) ON DELETE CASCADE
    , CONSTRAINT no_dup_book_author UNIQUE (author_id, book_id)
);
\end{lstlisting}
\end{block}
\end{frame}

\subsection{Features}

\begin{frame}[fragile]{}
\begin{block}{''UPSERT''}
\scriptsize
\begin{lstlisting}[language=SQL,inputencoding=latin9,escapeinside={(*@}{@*)}]
 MERGE INTO tablename USING table_reference ON (condition)
   WHEN MATCHED THEN
   UPDATE SET column1 = value1 [, column2 = value2 ...]
   WHEN NOT MATCHED THEN
   INSERT (column1 [, column2 ...]) VALUES (value1 [, value2 ...]);
\end{lstlisting}
\end{block}

\begin{block}<2->{SQLite}
\begin{itemize}
\item<2-> Unsupported by SQLite
\item<3-> \lstinline[language=SQL,inputencoding=latin9]!INSERT OR REPLACE! deletes before insert
\item[$\rightarrow$]<4-> Kills \texttt{UPDATE} Trigger
\end{itemize}
\end{block}
\end{frame}

\begin{frame}[fragile]{}
\begin{block}{Perl helps out}
\scriptsize
\begin{lstlisting}[language=Perl,inputencoding=latin9,escapeinside={(*@}{@*)}]
{
    my $goethe_id = $self->cmdb->upsert( "author",
      { author_name => "Johann Wolfgang von Goethe",
        author_birthday => "28 August 1749",
	author_died_at => "22 March 1832" } )->[0]->[0];
    my $faust_id = $self->cmdb->upsert( "book",
      { book_name => "Faust. Der Tragödie erster Teil" } )->[0]->[0];
    my $faust_author => $self->cmdb->upsert( "book_athors",
      { author_id => $goethe_id, book_id => $faust_id } )->[]->[];
}
\end{lstlisting}
\end{block}

\begin{block}<2->{}
\begin{itemize}
\item Limited to CMDB
\item<3-> Refuse updates of identifying columns (UNIQUE constraints)
\item<4-> WHERE clause derived from UNIQUE constraints
\end{itemize}
\end{block}
\end{frame}

\subsection{Meanwhile II}

\begin{frame}[fragile]{CMDB to Hiera}
\begin{block}{YAML Generator}
\begin{itemize}
\item Development team read via \textit{RESTful API} the theoretical configuration set
\item<2-> \textit{Hiera} \texttt{YAML} files are written
\item<3-> additional exports are managed via \textit{Hiera}
\item<4-> Puppet classes are rewritten to understand new ER model
\end{itemize}
\end{block}
\end{frame}

%  “You're entirely bonkers. But I'll tell you a secret: All the best people are.”
% ― Lewis Carroll, Alice in Wonderland

%\begin{frame}[fragile]{}
%\begin{block}{}
%\end{block}
%\end{frame}

\part{Finish}

\section{Goals reached}

\begin{frame}[fragile]{}
\begin{block}{Goals reached}
\begin{itemize}
\item[$\rightarrow$] Circle closed
\item[$\rightarrow$]<2-> actual-theoretical comparison done via processing scan database
\item[$\rightarrow$]<3-> unmaintainted installation via cronjob possible
\item[$\rightarrow$]<4-> reaction in emergency cases by organized component moving done multiple times
\item[$\rightarrow$]<5-> monitoring, sensors, alarming open
\end{itemize}
\end{block}
\end{frame}

\section{Conclusion}

\begin{frame}[fragile]{Conclusion}
\begin{block}{Can a programming language save a life}
\begin{itemize}
\item Yes, it can - but here is saves our business
\end{itemize}
\end{block}
\end{frame}

\section{Thank you}

\subsection{Hiring}

\begin{frame}[fragile]{We are hiring}
\begin{block}{New Job?}
Ask me for details
\end{block}
\end{frame}

\subsection{Thank you}

% THANKS
\begin{frame}[fragile]{Thank You For Listening}
\begin{block}{Questions?}
Jens Rehsack \textless{}\href{mailto:rehsack@cpan.org}{rehsack@cpan.org}\textgreater{} \\
Cologne
\end{block}
\end{frame}

\end{document}
