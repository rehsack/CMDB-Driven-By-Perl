\documentclass[ngerman,xcolor={table,dvipsnames},smaller,compress,hyperref={bookmarks,colorlinks}]{beamer}%,handout

\usepackage{url}
\usepackage{listings}
\usepackage[latin9]{inputenc}
\usepackage{textcomp}
%\usepackage{auto-pst-pdf}
\usepackage{graphicx}
\usepackage{pstricks}
\usepackage{pst-node}
\usepackage{pst-uml}
\usepackage{pst-tree}
\usepackage{pst-blur}

\usepackage{amsfonts}
\usepackage{amssymb}
\usepackage{amsmath}
\usepackage{pifont}

% for multipage tables (xtab or longtable
\usepackage{longtable}
\usepackage{lscape}
\usepackage{booktabs}
\usepackage[tableposition=top]{caption}

\title{CMDB Driven by Perl}
\subtitle{Road to a Perl "driven" Configuration Management Database}
\author{Jens Rehsack}
\institute[Niederrhein.PM]{Niederrhein Perl Mongers}
\date{The Perl Conference in Amsterdam}

\usetheme[secheader]{Boadilla}
\setbeamertemplate{navigation symbols}{}

\newcommand{\perlfilename}[1]{{\color{cyan}{\textit{\begingroup \urlstyle{sf}\Url{#1}}}}}
\newcommand{\xsfilename}[1]{{\color{olive}{\textit{\begingroup \urlstyle{sf}\Url{#1}}}}}
\newcommand{\hfilename}[1]{{\color{olive}{\textit{\begingroup \urlstyle{sf}\Url{#1}}}}}
\newcommand{\cfilename}[1]{{\color{magenta}{\textit{\begingroup \urlstyle{sf}\Url{#1}}}}}
\newcommand{\variable}[1]{{\color{violet}{\textsf{#1}}}}
\newcommand{\variablew}[1]{{{\textsf{#1}}}}
\newcommand{\command}[1]{'\texttt{#1}'}

\def\signed #1{{\leavevmode\unskip\nobreak\hfil\penalty50\hskip2em
   \hbox{}\nobreak\hfil(#1)%
   \parfillskip=0pt \finalhyphendemerits=0 \endgraf}}
 
\newsavebox\mybox
\newenvironment{aquote}[1]
   {\savebox\mybox{#1}\begin{quotation}}
   {\signed{\usebox\mybox}\end{quotation}}

\makeatletter
\makeatother

\begin{document}

\psset{angleA=90,angleB=-90}
\lstset{language=Perl,
        basicstyle=\ttfamily,
        keywordstyle=\color{Maroon},
        commentstyle=\color{Blue}, 
        stringstyle=\color{Green},
        showstringspaces=false}

\AtBeginPart{\begin{frame}<beamer> \frametitle{Overview} \partpage \tableofcontents[current] \end{frame}}

\frame{\maketitle}

\part{Introduction}

\section{Introduction}

\subsection{Motivation}

\begin{frame}[fragile]{Motivation}
\begin{block}{Progress}
\begin{aquote}{Robert A. Heinlein}
Progress isn't made by early risers. It's made by lazy men trying to find easier ways to do something. \\
\end{aquote}
\end{block}
\end{frame}

\begin{frame}[fragile]{Motivation}
\begin{block}{Efficiency}
\begin{aquote}{Elizabeth Mattijsen}
Business success is because of Perl. It enables us to deliver right solutions in days instead of months. \\
\end{aquote}
\end{block}
\end{frame}

\begin{frame}[fragile]{Goal}
\begin{block}{Automation}
Full flavoured systems management
\begin{itemize}
\item<2-> Installation without Administrator interaction
\item<3-> Control sensors and alarming
\item<4-> Ensured system state by actual-theoretical comparison
\item<5-> Faster reaction in emergency cases by organized component moving
\item<6-> Have an up-to-date ''Operation Handbook'' as well as archiving them
\end{itemize}
\end{block}
\end{frame}

\part{Challenge}

\section{Beginning}

\subsection{Taking over}

\begin{frame}[fragile]{Begin with reporting}
\begin{block}{Begin with reporting}
In the beginning was the (Installation-)Report
\begin{itemize}
\item<2-> Technical Sales defined an XML Document for Change Requests and Status Reports
\item<3-> Based on work of forerunner a 70\% solution could be delivered
\item<4-> Document Definition lacks entity-relations
\item<5-> Document Definition misses technolgy requirements
\item[$\Rightarrow$]<6-> Appears to be a dead end
\end{itemize}
\end{block}
\end{frame}

\subsection{After reporting}

\begin{frame}[fragile]{From reporting to \ldots}
\begin{block}{Mind the goal}
\begin{aquote}{Lewis Carroll}
Alice: Would you tell me, please, which way I ought to go from here? \\
The Cheshire Cat: That depends a good deal on where you want to get to. \\
\end{aquote}
\end{block}
\end{frame}

\subsection{Mine Sweeper}

\begin{frame}[fragile]{Where do I begin}
\begin{block}{To write the workflow how great Perl 5 can be}
\begin{itemize}
\item The project was in a state where a developer created a particular Report based on the existing snapshot.
\item<2-> This solution did not maintain an abstraction layer for gathered data - every time when the report needs an extension, an end-to-end (snapshot to XML-Tag) enhancement had to be created.
\item<3-> Changes shall be deployed from the same report format as installations are reported.
\item<4-> We have to be able to say at any moment what is operated on the platform.
\end{itemize}
\end{block}
\end{frame}

\subsection{Baby Steps}

\begin{frame}[fragile]{Baby Steps}
\begin{block}<1->{Improve knowledge}
Based on identified issues the first goal had to be to identify all entities and their relations together
\end{block}
\begin{block}<2->{Surrounded}
Problem: The entire platform was completely unstructured
\end{block}
\end{frame}

\begin{frame}[fragile]{Baby Steps}
\begin{block}{Multiple Beginnings}
\begin{itemize}
\item The already known ''(Installation-)Report''
\item<2-> Platform Snapshot (SCM Repository of selected configuration files)
\item<3-> Puppet Classes (without \textit{Hiera}) mixed with \textit{Configuration Items} (within \textit{Hiera}) and prepared configuration files (unsupervised)
\begin{description}[Hiera]
\item[Hiera]<4-> is Puppet's built-in key/value data lookup system. By default, it uses simple YAML or JSON files, although one can extend it to work with almost any data source.
\end{description}
\end{itemize}
\end{block}
\end{frame}

\part{Sweat}

\section{World Domination}

\subsection{Separation}

\begin{frame}[fragile]{Circle in the Sand}
\begin{block}{}

\resizebox{0.9\textwidth}{!}{
\begin{pspicture}(-7,-5)(7,5)%\psgrid
     % 
     \uput{4}[90](0,0){\rnode{platform}{\umlClass{\underline{:platform}}{}}}
     \uncover<2->{\uput{4}[18](0,0){\rnode{snapshot}{\umlClass{\underline{:snapshot}}{}}}}
     \uncover<3->{\uput{4}[306](0,0){\rnode{scandb}{\umlClass{\underline{:scan\_db}}{}}}}
     \uncover<4->{\uput{4}[234](0,0){\rnode{cmdb}{\umlClass{\underline{:cmdb}}{}}}}
     \uncover<5->{\uput{4}[162](0,0){\rnode{hiera}{\umlClass{\underline{:hiera}}{}}}}
     \uncover<8->{\rput(-6,-4){\rnode{report}{\umlClass{\underline{:report}}{}}}}

     \uncover<2->{\ncarc[arcangle=30,linewidth=0.5pt,linestyle=solid]{->}{platform}{snapshot}\naput*[Ynodesep=0.5]{collect}}
     \uncover<3->{\ncarc[arcangle=30,linewidth=0.5pt,linestyle=solid]{->}{snapshot}{scandb}\naput*[Ynodesep=0.5]{scan}}
     \uncover<4->{\ncarc[arcangle=30,linewidth=0.5pt,linestyle=solid]{->}{scandb}{cmdb}\naput*[Ynodesep=0.5]{process-scan}}
     \uncover<5->{\ncarc[arcangle=30,linewidth=0.5pt,linestyle=solid]{->}{cmdb}{hiera}\naput*[Ynodesep=0.5]{yaml-gen}}
     \uncover<6->{\ncarc[arcangle=30,linewidth=0.5pt,linestyle=solid]{->}{hiera}{platform}\naput*[Ynodesep=0.5]{puppet run}}
     \uncover<7->{\nccircle[angleA=-40,angleB=90,arm=.8,linearc=.2,linewidth=0.5pt,linestyle=solid]{->}{cmdb}{1cm}\naput*[Ynodesep=0.5]{CRQ}}
     \uncover<8->{\ncarc[arcangle=30,linewidth=0.5pt,linestyle=solid]{->}{cmdb}{report}\naput*[Ynodesep=0.5]{export}}

\end{pspicture}
}%end resizebox

\end{block}
\end{frame}

\section{Concerns}

\subsection{Identifying}

\begin{frame}[fragile]{Technical Concerns}
\begin{block}{Rough}
\begin{itemize}
\item Collecting platform parameters (to query them in structured way)
\item<2-> Identify coherences of \textit{Configuration Items} (CI)
\item<3-> Define a data model
\item<4-> Define technical requirements
\end{itemize}
\end{block}
\end{frame}

\begin{frame}[fragile]{Practical Concerns}
\begin{block}{Practical}
\begin{itemize}
\item Validity of \texttt{CI}'s
\item<2-> Limits of our \texttt{CI}'s
\item<3-> Data ownership of \texttt{CI}'s
\item<4-> Methods to persist \texttt{CI}'s
\item<5-> Methods to access \texttt{CI}'s
\item<6-> Permission management
\end{itemize}
\end{block}
\end{frame}

\section{Tuck In!}

\begin{frame}<beamer> \frametitle{Overview} \partpage \tableofcontents[current] \end{frame}

\subsection{MI:5}

\begin{frame}[fragile]{Impossible Things}
\begin{block}{Impossible Things}
\begin{aquote}{Lewis Carroll}
Alice laughed. ''There's no use trying,'' she said: ''one can't believe impossible things.'' \\
\hfil \\
''I daresay you haven't had much practice,'' said the Queen. ''When I was your age,
I always did it for half-an-hour a day. Why, sometimes I've believed as many as six
impossible things before breakfast.''
\end{aquote}
\end{block}
\end{frame}

\subsection{Home Improvement}

\begin{frame}[fragile]{The Fool with a Tool}
\begin{block}{Try again}
So we closed our eyes, took a deep breath (multiple times) and looked around for tools to store serialized data and read in structured way \ldots
\end{block}
\begin{block}<2->{Tool Time}
\begin{description}[Data Files]
\item[MongoDB]<2-> allows easy storing in any format - but lacks structured querying dedicated entities (configuration items)
\item[Data Files]<3-> delegate relationship handling completely to business logic
\item[AnyData2]<4-> gotcha - allows reading most confusing stuff and could be queried in structured way
\end{description}
\end{block}
\end{frame}

\begin{frame}[fragile]{\ldots is still a Fool}
\begin{block}{Volatile Structure}
\begin{itemize}
\item Persist structured data using SQLite
\item<2-> Define a data model representing existing relations
\item<3-> Develop \texttt{AnyData2::Format} classes representing defined ER (\textit{Entity Relationship}) model
\item<4-> Develop simple MOP inside this AnyData2 instance to manage attributes vs. columns
\item<5-> Glue everything together using SQL
\end{itemize}
\uncover<6->{The entire ER model remains a moving target}
\end{block}
\end{frame}

\subsection{Clean Picture}

\begin{frame}[fragile]{Abstraction Layer}
\begin{block}{\ldots of configured components}
\begin{itemize}
\item Focus the goal to know what is operated
\item<2-> Depth first search over all component configuration files
\item<3-> Identify relationships (remember: there is no operation model at all)
\item<4-> Clean up configuration when no reasonable relationships can be resolved or relationships are conflicting
\end{itemize}
\end{block}
\end{frame}

\subsection{Control}

\begin{frame}[fragile]{Moo in practice}
\begin{block}{Moo in practice}
It appears that the tools helping to do safe IoT device updating are the same tools helping to coordinate CI determining:
\begin{description}[\texttt{MooX::ConfigFromFile}]
\item[\texttt{MooX::Cmd}]<2-> helps separating concerns
\item[\texttt{MooX::ConfigFromFile}]<3-> helps contribute ''divine wisdom''
\item[\texttt{MooX::Options}]<4-> allow overriding ''divine wisdom'' by ''individual wisdom''
\item[\texttt{MooX::Log::Any}]<5-> feeds \texttt{DBIx::LogAny}
\end{description}
\end{block}
\end{frame}

\begin{frame}[fragile]{Moo in background}
\begin{block}{Moo in background}
\begin{itemize}
\item Manage database connections based on concerns
\item<2-> Manage \texttt{CI} structures based on relations
\item<3-> Manage Web-API integration
\end{itemize}
\end{block}
\end{frame}

\subsection{Reflecting Relationships}

\begin{frame}[fragile]{Craziness}
\begin{block}{Crazy}
\begin{aquote}{Lewis Carroll}
I'm not crazy. My reality is just different than yours. \\
\hfil \\
The Cheshire Cat \\
\end{aquote}
\end{block}
\end{frame}

\begin{frame}[fragile]{Harmonization}
\begin{block}{Harmonize Craziness}
\begin{itemize}
\item Practically any administrator had a different background regarding to the platform components thus a different picture of their relationships
\item<2-> EPIC battles leads to common craziness
\item<3-> ER model analysis sessions uncovered holes in picture
\end{itemize}
\end{block}
\end{frame}

\subsection{Meanwhile}

\begin{frame}[fragile]{Civilized}
\begin{block}{}
\begin{aquote}{Lewis Carroll}
\small
March Hare: Have some wine. \\
(Alice looked all round the table, but there was nothing on it but tea.) \\
Alice: I don't see any wine. \\
\hfil \\
March Hare: There isn't any. \\
Alice: Then it wasn't very civil of you to offer it. \\
March Hare: It wasn't very civil of you to sit down without being invited. \\
\end{aquote}
\end{block}
\end{frame}

\begin{frame}[fragile]{Adding a Goal}
\begin{block}{CentOS 5 ends its maintenance}
\begin{itemize}
\item Many of existing tools need to be upgraded
\item<2-> Upgraded tools don't support existing hacks anymore
\item<3-> Existing hacks must be replaced by a reasonable configuration structure
\item<4-> Same problem like the report format:
\begin{itemize}
\item<5-> neither the ER model of platform components nor issues of platform where known
\item<6-> nor cared about
\end{itemize}
\end{itemize}
\end{block}
\end{frame}

\begin{frame}[fragile]{Self Protection}
\begin{block}{Delegation}
We learned from mistakes of past:
\begin{itemize}
\item<2-> No responsibility taken for filling weird puppet templates
\item<3-> No external data will be managed
\item<4-> No precompiled/puzzled resources are prepared
\item[$\Rightarrow$]<5-> ER model of \texttt{CMDB} is presented via \textit{RESTful API}
\end{itemize}
\end{block}
\end{frame}

\subsection{Merging Pictures}

\begin{frame}[fragile]{}
\begin{block}{Scan completed}
\begin{itemize}
\item Early implementation of above mentioned \textit{RESTful API} run against \textit{ScanDB}
\item<2-> \textit{ScanDB} represents just a view of the configuration snapshot
\item<3-> There is no future\uncover<4->{, nor past}
\item<5-> Time for \textit{CMDB} to enter the stage
\end{itemize}
\end{block}
\end{frame}

\subsection{Future and Past}

\begin{frame}[fragile]{}
\begin{block}{Customers \ldots}
\scriptsize
\begin{lstlisting}[language=SQL,inputencoding=latin9,escapeinside={(*@}{@*)}]
CREATE TABLE customer_t
(
      customer(*@\pnode(0,0){A}{}@*)_id INTEGER PRIMARY KEY
    -- entity stuff
    , c(*@\pnode(0,0){B}{}@*)ustomer_name VARCHAR(80) UNIQUE NOT NULL
    -- cmdb stuff
    , valid(*@\pnode(0,0){C1}{}@*)_from DATETIME NOT NULL
    , valid(*@\pnode(0,0){C2}{}@*)_to DATETIME
    , modified(*@\pnode(0,0){C3}{}@*)_at DATETIME NOT NULL
    , modified(*@\pnode(0,0){C4}{}@*)_by VARCHAR(32) NOT NULL
);
\end{lstlisting}
\end{block}

\begin{itemize}
\item<2-> primary key and \rnode{a}{global identifier} for this data type
\item<3-> the \rnode{b}{payload} of this data type, automatically indexed
\item<4-> CMDB manages history and updates using \rnode{c}{these} columns
\end{itemize}
\uncover<2>{\nccurve[linecolor=blue]{->}{a}{A}}
\uncover<3>{\nccurve[linecolor=teal]{->}{b}{B}}
\uncover<4>{\nccurve[linecolor=violet]{->}{c}{C1}
            \nccurve[linecolor=violet]{->}{c}{C2}
	    \nccurve[linecolor=violet]{->}{c}{C3}
	    \nccurve[linecolor=violet]{->}{c}{C4}
}
\end{frame}

\begin{frame}[fragile]{}
\begin{block}{VPN links to customers \ldots}
\scriptsize
\begin{lstlisting}[language=SQL,inputencoding=latin9,escapeinside={(*@}{@*)}]
CREATE TABLE vpn_link_t
(
      vpn_link_id INTEGER PRIMARY KEY
    -- entity stuff
    , customer_i(*@\pnode(0,0){A}{}@*)d INTEGER NOT NULL
    , v(*@\pnode(0,0){B}{}@*)pn_link_type VARCHAR(12)
    , custom(*@\pnode(0,0){C1}{}@*)er_net VARCHAR(64) UNIQUE NOT NULL
    , servic(*@\pnode(0,0){C2}{}@*)es_net VARCHAR(64) UNIQUE NOT NULL
    -- cmdb stuff
    , valid_from DATETIME NOT NULL
    , valid_to DATETIME
    , modified_at DATETIME NOT NULL
    , modified_by VARCHAR(32) NOT NULL
    -- FK
    , FOREIGN KEY (customer_id) REFERENCES customer_t(customer_id)
      ON UPDATE CASCADE ON DELETE CASCADE
);
\end{lstlisting}
\end{block}

\begin{itemize}
\item<2-> refer the \rnode{a}{customer}
\item<3-> \rnode{b}{support} Cisco, Juniper, Paolo Alto, \ldots
\item<4-> networks must be unique or network \rnode{c}{admins kill} you
\end{itemize}
\uncover<2>{\nccurve[linecolor=teal]{->}{a}{A}}
\uncover<3>{\nccurve[linecolor=blue]{->}{b}{B}}
\uncover<4>{\nccurve[linecolor=blue]{->}{c}{C1}
            \nccurve[linecolor=blue]{->}{c}{C2}
}
\end{frame}

\begin{frame}[fragile]{}
\begin{block}{Moo Interception}
\scriptsize
\begin{lstlisting}[language=Perl,inputencoding=latin9,escapeinside={(*@}{@*)}]
package Foo::Role::Database::CMDB;
use Moo::Role(*@\pnode(0,0){A}{}@*);
requires "log";

has cmdb => (
    is       => "ro",
    required => 1,
    handles  => "Foo::Role::Database",
    isa      => sub {
        _INSTANCE_OF($_[0], "Foo::Helper::CMDB") and $_[0]->DOES("Foo::Role::Database")
          and return;
        die "Insufficient initialisation parameter for cmdb";
    },
    c(*@\pnode(0,0){B}{}@*)oerce => sub {
        _HASH($_[0]) and return Foo::Helper::CMDB->new(%{$_[0]});
        $_[0];
    },
);
\end{lstlisting}
\end{block}

\begin{itemize}
\item<2-> role can be \rnode{a}{consumed} by any class needing access to CMDB
\item<3-> \rnode{b}{transform} hash initializer into object
\end{itemize}
\uncover<2>{\nccurve[linecolor=olive]{->}{a}{A}}
\uncover<3>{\nccurve[linecolor=olive]{->}{b}{B}}
\end{frame}

\begin{frame}[fragile]{}
\begin{block}{Hard work}
\scriptsize
\begin{lstlisting}[language=Perl,inputencoding=latin9,escapeinside={(*@}{@*)}]
package Foo::Helper::CMDB;
use Moo; extends "Foo::Helper::DatabaseClass";
has config(*@\pnode(0,0){A}{}@*)_tables => (is => "lazy", ...);
has history_tabl(*@\pnode(0,0){B}{}@*)es => (is => "lazy", ...);
around deploy => sub { ...
my @tables = @{$self->config_tables};
foreach my $tbl (@tables) {
  my @h(*@\pnode(0,0){C}{}@*)ist_coldefs =
    map { my $default = defined $_->[4] ? " DEFAULT $_->[4]" : "";
      $pure_cols{$_->[1]} ? ("$_->[1] $_->[2]$default")
	: ("ol(*@\pnode(0,0){D}{}@*)d_$_->[1] $_->[2]$default", "n(*@\pnode(0,0){E}{}@*)ew_$_->[1] $_->[2]$default")
    } @table_info;
}
unshift @hist_coldefs, "${base_name}_hist_id INTEGER PRIMARY KEY";
my $hist_defs = join("\n    , ", @hist_coldefs);
my $hist_tbl = <<EOCHT;
CREAT(*@\pnode(0,0){F}{}@*)E TABLE ${base_name}_hist (
      ${hist_defs}
);
EOCHT
\end{lstlisting}
\end{block}

\begin{itemize}
\item<2-> that \rnode{a}{are} all tables with trailing \texttt{\_t} in their names
\item<3-> \rnode{f}{create} \rnode{b}{history} table for each config table
\item<4-> \rnode{c}{memoizing} \rnode{d}{old} and \rnode{e}{new} values on updating payload
\end{itemize}
\uncover<2>{\nccurve[linecolor=blue]{->}{a}{A}}
\uncover<3>{\nccurve[linecolor=olive]{->}{b}{B}
            \nccurve[linecolor=olive]{->}{f}{F}}
\uncover<4>{\nccurve[linecolor=violet]{->}{c}{C}
            \nccurve[linecolor=violet]{->}{d}{D}
	    \nccurve[linecolor=violet]{->}{e}{E}}
\end{frame}

\begin{frame}[fragile]{}
\begin{block}{Hard work (continued) - INSERT}
\scriptsize
\begin{lstlisting}[language=Perl,inputencoding=latin9,escapeinside={(*@}{@*)}]
my $new_cols = join(", ", map { $pure_cols{$_} ? $_ : "new_$_" }
               grep { !$skipped{$_} } @table_cols);
my $new_vals = join(", ", map {"NEW.$_"} grep { !$skipped{$_} } @table_cols);

my $new_trgr = <<EONT;
CREATE TRIGGER new_${base_name}_row AFTER INSERT ON ${base_name}_t
BEGIN
  INSERT INTO ${base_name}_hist ($new_cols)
  VALUES ($new_vals);
END;
EONT
\end{lstlisting}
\end{block}


\begin{itemize}
\item[$\Rightarrow$]<2-> ON INSERT fill history rows without filling "OLD\_" columns
\end{itemize}
\end{frame}

\begin{frame}[fragile]{}
\begin{block}{Hard work (continued) - UPDATE}
\scriptsize
\begin{lstlisting}[language=Perl,inputencoding=latin9,escapeinside={(*@}{@*)}]
my (@updt_cols, @rfrs_cond, @updt_vals);
foreach my $colnm (grep { !$skipped{$_} } @table_cols) {
  my @lcd = $pure_cols{$colnm} ? $colnm : ("old_${colnm}", "new_${colnm}");
  push @updt_cols, @lcd;
  push @rfrs_cond, $pure_cols{$colnm}
    ? _cmp_if_nullable("existing.${colnm}", "NEW.${colnm}")
    : (_cmp_if_nullable("existing.old_${colnm}", "OLD.${colnm}"),
       _cmp_if_nullable("existing.new_${colnm}", "NEW.${colnm}"));
  push @updt_vals, $pure_cols{$colnm}?("NEW.$colnm"):("OLD.$colnm","NEW.$colnm");
}
my $updt_cols = join(", ", @updt_cols);
my $updt_refreshed_cols = join(", ", map { "refreshed.$_" } @updt_cols);
my $updt_vals = join(", ", @updt_vals);
\end{lstlisting}
\end{block}

\begin{itemize}
\item[$\Rightarrow$]<2-> Prepare for a bit complexer trigger
\item[$\Rightarrow$]<3-> Distinguish between real updates and just ''touches''
\end{itemize}

\end{frame}

\begin{frame}[fragile]{}
\begin{block}{Hard work (continued) - UPDATE}
\scriptsize
\begin{lstlisting}[language=SQL,inputencoding=latin9,escapeinside={(*@}{@*)}]
my $updt_trgr = <<EONT;
CREATE TRIGGER updt_${base_name}_row AFTER UPDATE ON ${base_name}_t
BEGIN
  INSERT OR REPLACE INTO ${base_name}_hist (${base_name}_hist_id, $updt_cols)
  VALUES (
    (SELECT MAX(existing.${base_name}_hist_id) ${base_name}_hist_id
     FROM ${base_name}_hist existing WHERE $updt_refreshed_cond),
    $updt_vals);
END;
EONT
\end{lstlisting}
\end{block}

\begin{itemize}
\item[$\Rightarrow$]<2-> ON UPDATE create history (\texttt{INSERT}) rows with "OLD\_"  and "NEW\_" columns \\
\uncover<3->{ except nothing changes (\texttt{REPLACE})}
\end{itemize}
\end{frame}

\subsection{Features}

\begin{frame}[fragile]{}
\begin{block}{''UPSERT''}
\scriptsize
\begin{lstlisting}[language=SQL,inputencoding=latin9,escapeinside={(*@}{@*)}]
 MERGE INTO tablename USING table_reference ON (condition)
   WHEN MATCHED THEN
   UPDATE SET column1 = value1 [, column2 = value2 ...]
   WHEN NOT MATCHED THEN
   INSERT (column1 [, column2 ...]) VALUES (value1 [, value2 ...]);
\end{lstlisting}
\end{block}

\begin{block}<2->{SQLite}
\begin{itemize}
\item<2-> Unsupported by SQLite
\item<3-> \lstinline[language=SQL,inputencoding=latin9]!INSERT OR REPLACE! deletes before insert
\item[$\rightarrow$]<4-> Kills \texttt{UPDATE} Trigger
\end{itemize}
\end{block}
\end{frame}

\begin{frame}[fragile]{}
\begin{block}{Perl helps out}
\scriptsize
\begin{lstlisting}[language=Perl,inputencoding=latin9,escapeinside={(*@}{@*)}]
  $self->cmdb->upsert( customer_t => {
    customer_name => "Foo Enterprises", });
  $self->cmdb->upsert( vpn_link_t => {
    customer_name => "Foo Enterprises",
    vpn_link_type => "Juniper",
    customer_net  => "10.116.47.8/29",
    services_net  => "10.126.47.8/29" } );
\end{lstlisting}
\end{block}

\begin{block}{SQL created \ldots}
\scriptsize
\begin{lstlisting}[language=SQL,inputencoding=latin9,escapeinside={(*@}{@*)}]
INSERT OR IGNORE INTO vpn_link_t (
  customer_id, vpn_link_type, customer_net, services_net, modified_by
) VALUES (
  (SELECT customer_id FROM customer_t WHERE customer_name=?),
  ?, ?, ?, ?);
UPDATE vpn_link_t SET vpn_link_type=?, customer_net=?, services_net=?,
  modified_by=?, touched_at=CURRENT_TIMESTAMP
WHERE changes()=0 AND customer_id=(
  SELECT customer_id FROM customer_t WHERE customer_name=?);
\end{lstlisting}
\end{block}
\end{frame}

\begin{frame}[fragile]{}
\begin{block}{Known limitations}
\begin{itemize}
\item Restricted to CMDB
\item<2-> Refuse updates of identifying columns (UNIQUE constraints)
\item<3-> WHERE clause derived from UNIQUE constraints
\end{itemize}
\end{block}
\end{frame}

\subsection{Meanwhile II}

\begin{frame}[fragile]{CMDB to Hiera}
\begin{block}{YAML Generator}
\begin{itemize}
\item Development team read via \textit{RESTful API} the theoretical configuration set
\item<2-> \textit{Hiera} \texttt{YAML} files are written
\item<3-> Additional exports are managed via \textit{Hiera}
\item<4-> Puppet classes are rewritten to understand new ER model
\end{itemize}
\end{block}
\end{frame}

%  “You're entirely bonkers. But I'll tell you a secret: All the best people are.”
% ― Lewis Carroll, Alice in Wonderland

%\begin{frame}[fragile]{}
%\begin{block}{}
%\end{block}
%\end{frame}

\part{Finish}

\section{Goals reached}

\subsection{Circle closed}

\begin{frame}[fragile]{Circle closed}
\begin{block}{}

\resizebox{0.9\textwidth}{!}{
\begin{pspicture}(-7,-5)(7,5)%\psgrid
     % 
     \uput{4}[90](0,0){\rnode{platform}{\umlClass{\underline{:platform}}{}}}
     \uput{4}[18](0,0){\rnode{snapshot}{\umlClass{\underline{:snapshot}}{}}}
     \uput{4}[306](0,0){\rnode{scandb}{\umlClass{\underline{:scan\_db}}{}}}
     \uput{4}[234](0,0){\rnode{cmdb}{\umlClass{\underline{:cmdb}}{}}}
     \uput{4}[162](0,0){\rnode{hiera}{\umlClass{\underline{:hiera}}{}}}
     \rput(-6,-4){\rnode{report}{\umlClass{\underline{:report}}{}}}

     \ncarc[arcangle=30,linewidth=0.5pt,linestyle=solid]{->}{platform}{snapshot}\naput*[Ynodesep=0.5]{collect}
     \ncarc[arcangle=30,linewidth=0.5pt,linestyle=solid,linecolor=red]{->}{snapshot}{scandb}\naput*[Ynodesep=0.5]{scan}
     \ncarc[arcangle=30,linewidth=0.5pt,linestyle=solid,linecolor=red]{->}{scandb}{cmdb}\naput*[Ynodesep=0.5]{process-scan}
     \ncarc[arcangle=30,linewidth=0.5pt,linestyle=solid,linecolor=red]{->}{cmdb}{hiera}\naput*[Ynodesep=0.5]{yaml-gen}
     \ncarc[arcangle=30,linewidth=0.5pt,linestyle=solid]{->}{hiera}{platform}\naput*[Ynodesep=0.5]{puppet run}
     \nccircle[angleA=-40,angleB=90,arm=.8,linearc=.2,linewidth=0.5pt,linestyle=solid,linecolor=red]{->}{cmdb}{1cm}\naput*[Ynodesep=0.5]{CRQ}
     \ncarc[arcangle=30,linewidth=0.5pt,linestyle=solid]{->}{cmdb}{report}\naput*[Ynodesep=0.5]{export}

     \uncover<2->{\ncarc[arcangle=30,linewidth=0.5pt,linestyle=solid,linecolor=blue]{->}{snapshot}{scandb}\naput*[Ynodesep=0.5]{scan}}
     \uncover<2->{\ncarc[arcangle=30,linewidth=0.5pt,linestyle=solid,linecolor=blue]{->}{scandb}{cmdb}\naput*[Ynodesep=0.5]{process-scan}}
     \uncover<2->{\ncarc[arcangle=30,linewidth=0.5pt,linestyle=solid,linecolor=blue]{->}{cmdb}{hiera}\naput*[Ynodesep=0.5]{yaml-gen}}
     \uncover<2->{\nccircle[angleA=-40,angleB=90,arm=.8,linearc=.2,linewidth=0.5pt,linestyle=solid,linecolor=cyan]{->}{cmdb}{1cm}\naput*[Ynodesep=0.5]{CRQ}}

\end{pspicture}
}%end resizebox

\end{block}
\end{frame}

\begin{frame}[fragile]{}
\begin{block}{Goals reached}
\begin{itemize}
\item[$\rightarrow$] Actual-theoretical comparison done via processing scan database
\item[$\rightarrow$]<2-> Unmaintainted installation via cronjob possible
\item[$\rightarrow$]<3-> Reaction in emergency cases by organized component moving done multiple times
\item[$\rightarrow$]<4-> Monitoring, sensors, alarming open
\end{itemize}
\end{block}
\end{frame}

\section{Conclusion}

\begin{frame}[fragile]{Conclusion}
\begin{block}{Can a programming language save a life}
\begin{itemize}
\item Yes, it can - but here it saves our business
\end{itemize}
\end{block}
\end{frame}

\section{Thank you}

\subsection{Thank you}

% THANKS
\begin{frame}[fragile]{Thank You For Listening}
\begin{block}{Questions?}
Jens Rehsack \textless{}\href{mailto:rehsack@cpan.org}{rehsack@cpan.org}\textgreater{} \\
Cologne
\end{block}
\end{frame}

\end{document}
